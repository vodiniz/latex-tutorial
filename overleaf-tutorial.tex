\documentclass[12pt, letterpaper]{report}
\usepackage[utf8]{inputenc}
\usepackage{graphicx}
\graphicspath{{images/}}

\title{First document}
\author{Vitor Oliveira Diniz\thanks{Tutorial made with overleaf help}}
\date{ }

\begin{document}
    \maketitle
    \tableofcontents

    \section*{Hello There
    }
    %Simple typing
    First Document. This is a simple example, with no extra parameters or packages included.
    %This line here is a comment. iT will not be printed in the document
    \bigbreak
    \
    %using bold italics and underlining
    Some of the greatest \textbf{discoveries} in \underline{science} were made by \textbf{\textit{accident}}.
    \bigbreak
    %emphasis with \emphaenglind
    Some of the greatest \emph{discoveries} in science were made by accident.
    \bigbreak
    
    \textit{Some of the greatest \emph{discoveries} in science were made by accident.}

    \bigbreak
    \textbf{Some of the greatest \emph{discoveries} in science were made by accident.}
    \bigbreak
    The mountain is immense, and it seems to be homogeneous in a large scale, everywhere we look at.
    \bigbreak


    \pagebreak
    \begin{figure}[h]
        \includegraphics[width=80mm]{mountains_wallpaper.png}
        \centering
        \caption{A nice mountain}
        \label{fig:mount1}
    \end{figure}

    As you can see in the figure \ref{fig:mount1} the function grows near 0. Also, in the page \pageref{fig:mount1} is the same example.

    \pagebreak

    \begin{itemize}
        \item The individual entries are indicated with a black dot, a so-called bullet.
        \item The text in the entries may be of any length.
    \end{itemize}

    \bigbreak

    \begin{enumerate}
        \item This is the first entry in our list
        \item The list numbers increase with each entry we add
    \end{enumerate}


    \pagebreak

    In physics\ the mass-energy equivalent is stated
    by the equation $E=mc^2$, discovered in 1905 by Albert Einstein.

    \bigbreak

    In physics\ the mass-energy equivalent is stated
    by the equation 
    \[E=mc^2\]
    discovered in 1905 by Albert Einstein. In natural units (c = 1), the formula expresses the identity 
    \begin{equation}
        E=m
    \end{equation}

    \pagebreak

    Subscripts in math mode are written as $a_b$ and superscripts are written as $a^b$.
    These can be combined an nested to write expressions such as 
    \[ T^{i_1 i_2 \dots i_p}_{j_1 j_2 \dots j_q} = T(x^{i_1}, dots, x^{i_p}, e_{j_1}, \dots, e_{j_q}) \]

    We write integrals using $\int$ and fractions using $\frac{a}{b}$. Limits are placed on integrals using superscripts and Subscripts:

    \[ \int_0^1 \frac{dx}{e^x} = \frac{e-1}{e} \]

    Lower case Greek letters are written as $\omega$ $\delta$ etc. 
    while upper case Greek letters are written as $\Omega$ $\Delta$
    
    Mathematical operators are prefixed with a backslash as 
    $\sin(\beta)$,
    $\cos(\alpha)$,
    $\log(x)$
    etc.


    \pagebreak

    \begin{abstract}
            This is a simple paragraph at the beggining of the document. A brief introcutions about the main subject.

    \end{abstract}

    Now that we have written our abstract, we can begin writting our first paragraph.
    

    This line will start a second Paragraph.


    \chapter{First Chapter}

    
        \section{Introduction}

            This is the first section.

            Mussum Ipsum, cacilds vidis litro abertis. Pra lá , depois divoltis porris, paradis.Diuretics paradis num copo é motivis de denguis.Praesent vel viverra nisi. Mauris aliquet nunc non turpis scelerisque, eget.Si u mundo tá muito paradis? Toma um mé que o mundo vai girarzis!

        \section{Second Section}

            Mussum Ipsum, cacilds vidis litro abertis. Pra lá , depois divoltis porris, paradis.Diuretics paradis num copo é motivis de denguis.Praesent vel viverra nisi. Mauris aliquet nunc non turpis scelerisque, eget.Si u mundo tá muito paradis? Toma um mé que o mundo vai girarzis!

            \subsection*{First Subsection}
                Mussum Ipsum, cacilds vidis litro abertis. Pra lá , depois divoltis porris, paradis.Diuretics paradis num copo é motivis de denguis.Praesent vel viverra nisi. Mauris aliquet nunc non turpis scelerisque, eget.Si u mundo tá muito paradis? Toma um mé que o mundo vai girarzis!

        \section*{Unnumbered Section}
            Mussum Ipsum, cacilds vidis litro abertis. Pra lá , depois divoltis porris, paradis.Diuretics paradis num copo é motivis de denguis.Praesent vel viverra nisi. Mauris aliquet nunc non turpis scelerisque, eget.Si u mundo tá muito paradis? Toma um mé que o mundo vai girarzis!

    \pagebreak


    \begin{center}
        \begin{tabular}{c c c}

            cell1 & cell2 & cell3 \\
            cell4 & cell5 & cell6 \\
            cell7 & cell8 & cell9
    
        \end{tabular}
    \end{center}

    \begin{center}
        \begin{tabular}{ |c|c|c| }
            \hline

                cell1 & cell2 & cell3 \\
                cell4 & cell5 & cell6 \\
                cell7 & cell8 & cell9 \\
            \hline
        \end{tabular}
    \end{center}


    \begin{center}
        \begin{tabular}{||c c c c||} 
            \hline
            Col1 & Col2 & Col2 & Col3 \\ [0.5ex] 
            \hline\hline
            1 & 6 & 87837 & 787 \\ 
            \hline
            2 & 7 & 78 & 5415 \\
            \hline
            3 & 545 & 778 & 7507 \\
            \hline
            4 & 545 & 18744 & 7560 \\
            \hline
            5 & 88 & 788 & 6344 \\ [1ex] 
            \hline
            \end{tabular}
       \end{center}

    Table \ref{table:data} is an example of referenced \LaTeX{} elements.

    \begin{table}[h!]
        \centering
        \begin{tabular}{||c c c c||} 
            \hline
            Col1 & Col2 & Col2 & Col3 \\ [0.5ex] 
            \hline\hline
            1 & 6 & 87837 & 787 \\ 
            2 & 7 & 78 & 5415 \\
            3 & 545 & 7& 5415 \\
            3 & 545 & 778 & 7507 \\
            4 & 545 & 18744 & 7560 \\
            5 & 88 & 788 & 6344 \\ [1ex] 
            \hline
        \end{tabular}
        \caption{Table to test captions and labels}
        \label{table:data}
    \end{table}





    

\end{document}
